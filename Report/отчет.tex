%% -*- coding: utf-8 -*-
\documentclass[12pt,a4paper]{scrartcl} 
\usepackage[utf8]{inputenc}
\usepackage[english,russian]{babel}
\usepackage{indentfirst}
\usepackage{misccorr}
\usepackage{graphicx}
\usepackage{amsmath}
\begin{document}
	\begin{titlepage}
		\begin{center}
			\large
			МИНИСТЕРСТВО НАУКИ И ВЫСШЕГО ОБРАЗОВАНИЯ РОССИЙСКОЙ ФЕДЕРАЦИИ
			
			Федеральное государственное бюджетное образовательное учреждение высшего образования
			
			\textbf{АДЫГЕЙСКИЙ ГОСУДАРСТВЕННЫЙ УНИВЕРСИТЕТ}
			\vspace{0.25cm}
			
			Инженерно-физический факультет
			
			Кафедра управления и информатики в технических системах
			\vfill
			
			\vfill
			
			\textsc{Отчет по практике}\\[5mm]
			
			{\LARGE Программаная реализация численного метода \textit{Решение системы линейных алгебраических уравнений методом Гаусса-Жордана.}}
			\bigskip
			
			1 курс, группа 1УТС
		\end{center}
		\vfill
		
		\newlength{\ML}
		\settowidth{\ML}{«\underline{\hspace{0.7cm}}» \underline{\hspace{2cm}}}
		\hfill\begin{minipage}{0.5\textwidth}
			Выполнил:\\
			\underline{\hspace{\ML}} А.\,А.~Ашла\\
			«\underline{\hspace{0.7cm}}» \underline{\hspace{2cm}} 2022 г.
		\end{minipage}%
		\bigskip
		
		\hfill\begin{minipage}{0.5\textwidth}
			Руководитель:\\
			\underline{\hspace{\ML}} С.\,В.~Теплоухов\\
			«\underline{\hspace{0.7cm}}» \underline{\hspace{2cm}} 2022 г.
		\end{minipage}%
		\vfill
		
		\begin{center}
			Майкоп, 2022 г.
		\end{center}
	\end{titlepage}

	\section{Введение}
	\label{sec:intro}
	
	% Что должно быть во введении
	\begin{enumerate}
		\item Текстовая формулировка задачи
		\item Пример кода, решающего данную задачу
		\item График
		\item Скриншот программы
	\end{enumerate}
	
	Пример приведен в пункте ~\ref{sec:exp} на стр.~\pageref{sec:exp}.
	
	\section{Ход работы}
	\label{sec:exp}
	
	\subsection{Код приложения}
	\label{sec:exp:code}
	\begin{verbatim}
		public static void main(String[] args) {
			int[] answers = {128, 208, 240, 154};
			int[][] matrix = {
				{12, 6, 2, 16},
				{20, 56, 18, 17},
				{18, 0, 34, 15},
				{2, 5, 17, 17}
			};
			SystemLinearAlgebraicEquations slau = new
			SystemLinearAlgebraicEquations(matrix, answers);
			slau.calculation();
			double[] result = slau.getResult();
			System.out.println("Ответ:");
			for(int i = 0; i < result.length; i++)
			System.out.printf("   x%d = %7.1f;\n", i + 1, result[i]);
		}
	
	    public class SystemLinearAlgebraicEquations {
	    	
	    	private static final String FORMAT = String.format("%s%d%s", "%", 8, ".1f");
	    	
	    	private final double[][] matrix;
	    	private final double[] answers;
	    	private final boolean[] lines;
	    	
	    	public SystemLinearAlgebraicEquations(int[][] matrix, int[] answers) {
	    		this.checkArguments(matrix, answers);
	    		this.matrix = parseMatrixIntToDouble(matrix);
	    		this.answers = parseIntToDouble(answers);
	    		this.lines = new boolean[answers.length];
	    	}
	    	
	    	private void checkArguments(int[][] matrix, int[] answers) {
	    		if(matrix == null || matrix.length == 0)
	    		throw new IllegalArgumentException("Matrix not may be null or empty.");
	    		if(answers == null || answers.length == 0)
	    		throw new IllegalArgumentException("Answers not may be null or empty.");
	    		if(matrix.length != answers.length)
	    		throw new IllegalArgumentException("Amount answer not equal height matrix.");
	    		if(!checkSquareMatrix(matrix))
	    		throw new IllegalArgumentException("Matrix not may be not square.");
	    	}
	    	
	    	private boolean checkSquareMatrix(int[][] matrix) {
	    		int size = matrix.length;
	    		for(int[] line : matrix) {
	    			if(line.length != size)
	    			return false;
	    		}
	    		return true;
	    	}
	    	
	    	private double[][] parseMatrixIntToDouble(int[][] matrix) {
	    		double[][] result = new double[matrix.length][];
	    		int count = 0;
	    		for(int[] line : matrix)
	    		result[count++] = parseIntToDouble(line);
	    		return result;
	    	}
	    	
	    	private double[] parseIntToDouble(int[] line) {
	    		double[] result = new double[line.length];
	    		for(int i = 0; i < line.length; i++)
	    		result[i] = line[i];
	    		return result;
	    	}
	    	
	    	public void calculation() {
	    		System.out.println("Дана СЛАУ:");
	    		printSLAU(matrix, answers);
	    		
	    		double element;
	    		int index;
	    		int row = 0;
	    		
	    		while(!isEnd(lines)) {
	    			System.out.println("####################### СТОЛБЕЦ №" + (row + 1)
	    			+ " #######################\n");
	    			
	    			System.out.println("Ищем разрешающий элемент в столбце #" + (row + 1) + ":");
	    			index = getIndexMin(matrix, lines, row);
	    			element = matrix[index][row];
	    			System.out.printf("элемент = %.1f;\n", element);
	    			
	    			System.out.println("\nДелим строку #" + (index + 1) + " на " + element + ":");
	    			divToElement(matrix[index], element);
	    			answers[index] /= element;
	    			printSLAU(matrix, answers);
	    			
	    			System.out.println("Обнулим в столбце #" + (row + 1) + " все элементы,
	    			кроме разрешающего:");
	    			toNullifyElements(matrix, answers, index, row);
	    			printSLAU(matrix, answers);
	    			
	    			lines[index] = true;
	    			row++;
	    		}
	    		System.out.println("########################################
	    		###################\n");
	    	}
	    	
	    	public double[] getResult() {
	    		if(matrix == null || matrix.length == 0 ||
	    	    answers == null || answers.length == 0)
	    		throw new IllegalArgumentException("Calculation failed.");
	    		return getAnswer(matrix, answers);
	    	}
	    	
	    	private double[] getAnswer(double[][] matrix, double[] answers) {
	    		double[] result = new double[answers.length];
	    		for(int i = 0; i < matrix.length; i++) {
	    			for(int j = 0; j < matrix[i].length; j++) {
	    				if(matrix[i][j] == 1.) {
	    					result[j] = answers[i];
	    				}
	    			}
	    		}
	    		return result;
	    	}
	    	
	    	private boolean isEnd(boolean[] lines) {
	    		for(boolean index : lines) {
	    			if(!index)
	    			return false;
	    		}
	    		return true;
	    	}
	    	
	    	private void toNullifyElements(double[][] matrix, double[] answers,
	    	int exceptLine, int row) {
	    		for(int i = 0; i < matrix.length; i++) {
	    			if(i == exceptLine)
	    			continue;
	    			double first = matrix[i][row];
	    			for(int j = 0; j < matrix[i].length; j++) {
	    				if(matrix[i][j] != 0.)
	    				matrix[i][j] -= first * matrix[exceptLine][j];
	    			}
	    			if(answers[i] != 0.)
	    			answers[i] -= first * answers[exceptLine];
	    		}
	    	}
	    	
	    	private void divToElement(double[] line, double div) {
	    		for(int index = 0; index < line.length; index++) {
	    			if(line[index] != 0.)
	    			line[index] /= div;
	    		}
	    	}
	    	
	    	private void printSLAU(double[][] matrix, double[] answers) {
	    		StringBuilder sb = new StringBuilder();
	    		for(int i = 0; i < answers.length; i++) {
	    			sb.append("|");
	    			for(int j = 0; j < matrix[i].length; j++)
	    			sb.append(" ").append(String.format(FORMAT, matrix[i][j]));
	    			sb.append(" | ").append(String.format(FORMAT, answers[i])).append(" |\n");
	    		}
	    		System.out.println(sb);
	    	}
	    	
	    	private int getIndexMin(double[][] matrix, boolean[] lines, int index) {
	    		int res = -1;
	    		double min = Integer.MAX_VALUE;
	    		for(int i = 0; i < matrix.length; i++) {
	    			if(!lines[i] && Math.abs(matrix[i][index]) < Math.abs(min) &&
	    			matrix[i][index] != 0) {
	    				res = i;
	    				min = matrix[i][index];
	    			}
	    		}
	    		return res;
	    	}
	    }
	    
	\end{verbatim}
	
	\subsection{Пример формулы}
	\label{sec:mathexample}
	
	Решение квадратных систем линейного алгебраического уравнения:
	\begin{equation}\label{eq:solv}
		A=\begin{pmatrix}
			a11 & a12 & \cdots  & a1n \\
			a21 & a22 & \cdots  & a2n \\
			\vdots  & \cdots  & \ddots  & \vdots  \\
			an1 & an2 & \cdots & ann 
		\end{pmatrix}
	    \hspace{0.5cm}
	    aii\neq 0
	    \hspace{0.5cm}
	    I=\begin{pmatrix}
	    	1 & 0 & \cdots & 0 \\
	    	0 & 1 & \cdots & 0 \\
	    	\vdots & \vdots & \ddots & \vdots \\
	    	0 & 0 & \cdots & 1
	    \end{pmatrix}
	\end{equation}
	
	Можно сослаться на уравнение~\eqref{eq:solv}.
	
\section{Пример вставки изображения}
\label{sec:picexample}
\begin{figure}[h]
	\centering
	\includegraphics[width=0.4\textwidth]{collage.png}
	\caption{Работа программы}\label{fig:par}
\end{figure}
	
	\section{Пример библиографических ссылок}
    http://geo.phys.spbu.ru/LDUS/files/books/LaTeX/LaTeX-Lvovsky.pdf
	
	\begin{thebibliography}{9}
		\bibitem{Knuth-2003}Кнут Д.Э. Всё про \TeX. \newblock --- Москва: Изд. Вильямс, 2003 г. 550~с.
		\bibitem{Lvovsky-2003}Львовский С.М. Набор и верстка в системе \LaTeX{}. \newblock --- 3-е издание, исправленное и дополненное, 2003 г.
		\bibitem{Voroncov-2005}Воронцов К.В. \LaTeX{} в примерах. 2005 г.
	\end{thebibliography}
\end{document}